\chapter{Background}

\comments{This is a narrative description of the general context within which your project fits. Depending on your particular project characteristics, you are required to include discussion of any or all of the following – previous related work; the work or objectives of a client; the essential principles of systems or techniques you are using.
All this narrative should be properly referenced to source material citations. Remember that a high class project will refer to background sources beyond just those on the Web. 
In writing this section you should pay close attention to your audience and their prior knowledge of the subject(s) that you are discussing. You should assume that your reader is a student who has just completed the second year of your degree programme and can therefore assume that the reader is familiar with all topics taught up to the end of the second year. Anything that is needed to understand your project and its context but which has not been taught by the end of the second year of your degree should be discussed in this background section. 
This section may include one or more of the following subsections. It is difficult to give prescriptive guidance on which subsections you should include as this depends on the nature of the project you are undertaking – you should discuss this with your supervisor.}

\comments{
\textbf{Problem Context:}
If your project involves significant work on a non-computing topic that is likely to be unfamiliar to most readers (e.g. linguistics, fluid dynamics) you should describe the important principles, concepts and terminology of that subject area in some detail. You will have had to learn these yourself in getting to grips with this unfamiliar topic and you should summarize what you learnt to enable the reader to understand your subsequent discussion on your project work and how this relates to the wider topic area.}

\comments{
\textbf{Comparison of Technologies:}
If there are several possible technologies that could be used in your project work you should present a comparative analysis and critical appraisal of each of these technologies. You should create a subsection for each of the technologies you discuss and title each subsection with the name of the technology it describes (e.g. object-oriented databases, XML ). Within each subsection you should provide an overview of the technology, its key features and its strengths and weaknesses in relation to your project.}

\comments{
\textbf{Alternative Solutions:}
If others have produced solutions or addressed similar problems to those addressed by your project you should describe those alternative solutions here. Similarly, if several possible approaches suggest themselves as ways of solving the problems inherent in your project you should discuss those here. You should provide a comparative analysis and critical appraisal of each alternative solution approach or existing solution, identifying their key features and their strengths and weaknesses in relation to your project.}

\comments{
\textbf{Comparison of Algorithms:}
There may be a number of different algorithms which could be applied to the central problems in your project and you have had to choose which of these algorithms are the most appropriate for your implementation. If this is the case then you should provide a comparative analysis and critical appraisal of each of the potentially applicable algorithms, highlighting their key features and their strengths and weaknesses in relation to your project.
}

\comments{
\textbf{Processes and Methodology:}
If your project is concerned with improving, implementing or evaluating a particular technical process or method of working you should discuss these in detail. We are not expecting you to describe what software development methodology you are following in implementing your project and you certainly do not need to regurgitate textbook descriptions of the Waterfall method here as everyone already knows that model well.
}

\comments{
\textbf{Referencing:}
The current template provides a Bibtex style file (\texttt{hull.bst}) that conforms to the University's requirements of using Harvard referencing style \citep{Hull2016HarvardRef}. Your references should be collected as Bibtex entries into the file \texttt{report.bib}. Below are some examples of citing these Bibtex entries:
\begin{itemize}
\item \citet{Kazhdan2006} developed a technique for creating watertight surfaces from oriented point samples acquired with 3D range scanners.
\item Thorough coverage of the new C++11 standard can be found in \citep{Stroustrup2013}.
\item Vulkan is a cross-platform 3D graphics and compute API \citep{Vulkan2016}.
\item \citet{Canny1986} proposed a computational approach to detecting edges in images.
\end{itemize}
Many online digital libraries, such as those from ACM\footnote{\url{http://dl.acm.org/}} and IEEE\footnote{\url{http://ieeexplore.ieee.org/}}, provide Bibtex data for their papers. Google Scholar\footnote{\url{https://scholar.google.com}} is also an easy way to search for academic publications and their Bibtex data.
}